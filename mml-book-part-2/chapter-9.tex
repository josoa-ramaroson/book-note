chapter{Linear Regression}
Find a function that maps inputs $x \in \mathbb{R}^D$ to corresopnding function values $f(x) \in \mathbb{R}$. \\ 
We assume zero-mean Gaussian noise.\\
\textbf{Requiered problem to be solved when finding a regression function:}
\begin{itemize}
	\item Choice of the model (type) and the parametrization of the regression function
	\item Finding good parameters
	\item Overfitting and model selection
	\item Relationship between loss functions and parameter priors
	\item Uncertainty modeling
\end{itemize}

\section{Problem Formulation}
Model the noise using a likelihood function:
\[
p(y|x) = \mathcal{N}(y|f(x), \sigma^2) \tag{9.1}
\]
Here, $x \in \mathbb{R}^D$ are inputs and $y \in \mathbb{R}$ are noisy function values (targets). \\
The relationship between $x$ and $y$ :
\[
y = f(x) + \upsilon \tag{9.2} 
\]
Where $\upsilon \sim \mathcal{N}(0, \sigma^2)$ is independent, identically distributed (i.i.d)  Gaussian measurement noise with mean 0 and variance $\sigma^2$.\\
\textbf{Objective :} to find a function that is close (similar) to the unknown function $f$ that generated the data and that generalizes well. \\
\textbf{Parametric models} (parameters $\theta$): $\sigma^2$ as the noise variance, we will focus on learning the model parameters $\theta$.\\
$\theta$ appear linearly in our model like:
\[
p(y|x, \theta) = \mathcal{N}(y|x^T  \theta, \sigma^2) \tag{9.3}
\]
\[
\Leftrightarrow y = x^T \theta + \epsilon, \; \; \epsilon \sim \mathcal{N} (0, \sigma^2) \tag{9.4}
\]
Where $\theta \in \mathbb{R}^D$ the parameters we seek.\\
The likelihood in (9.3) is the probability density function of $y$ evaluated at $x^{T} \theta$.\\ \\ 
A Dirac delta (delta function) is zero everywhere except at a single point, and its integral is $1$. It can be considered a Gaussian in the limit of $\sigma^2 \rightarrow 0$ likelihood
\textbf{Example 9.1} see p.291\\ \\
Linear regression refers to models that are linear in the parameters. \\
Linear regression model $\Rightarrow$ linear input $x$.\\
For non linear input $\Rightarrow$ $y = \Phi^{T}(x) \theta$ where $\Phi$ is also linear regression model.\\
From here, we assume that the noise variance $\sigma^2$ is known.

\section{Parameter Estimation}